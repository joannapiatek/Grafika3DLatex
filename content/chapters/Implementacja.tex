\section{Implementacja projektu}

\subsection{Definicja sterowania}

\subsection{Sterowanie postacią}
Sterowanie postacią zostało zaimportowane z Standard Assets zawierających m.in. moduł FirstPersonCharacter, w którym zostało zdefiniowane całe sterowanie postacią: kiwanie się kamery przy chodzeniu, dźwięki kroków, postać w kształcie kapsuły, itp.

\subsection{Poruszanie kijem}

\subsection{Piłka}

\subsection{Wazon i jego rozbicie}
W celu zademonstrowania możliwości interakcji i destrukcji otoczenia, w pomieszczeniu umieszczona została waza. Przy mocnym uderzeniu, upadku, rozpada się na mniejsze fragmenty, z którymi użytkownik może przeprowadzać dalszą interakcję.

\subsubsection{Elementy}
\paragraph{Waza}
Waza została zamodelowania w środowisku \textit{Blender}. Podstawowym kształtem dla wazy jest okrąg, z którego za pomocą narzędzia \textit{Extrude} i skalowania w prosty sposób można utworzyć symetryczne, owalne przedmioty. \\
Krawędzie utworzonego modelu zostały wygładzone za pomocą narzędzia \textit{Smooth} dostępnego w oprogramowaniu. 
\paragraph{Fragmenty}
Modele rozbitej wazy - zarówno całej, jak i już rozbitych elementów które podlegają dalszej destrukcji zostały również utworzone w narzędziu \textit{blender}.
W tym celu skorzystano z dodatkowego narzędzia \textit{Cell Fracture} dostępnego dla \textit{blendera}. Dodatek ten pozwala na podział utworzonego modelu na  \textit{n} fragmentów za pomocą tesselacji Woronoja. Środki obszarów mogą zostać wygenerowane automatycznie, lub zostać wybrane przez użytkownika. Korzystając z narzędzia \textit{Cell Fracture} użytkownik ma do dyspozycji szereg opcji, m.in. wybór maksymalnej liczby fragmentów na które zostanie podzielony model, czy wielkość odstępu pomiędzy nowo utworzonymi fragmentami. Z jednej strony pogarsza to wizualne wrażenia przy rozbiciu przedmiotu, jednakże pozwala ograniczyć ryzyko "wybuchu" szczątków na wskutek błędu silnika fizycznego. Ma to szczególnie duże znaczenie przy podziale modelu na kilkadziesiąt lub kilkaset fragmentów.

\subsubsection{Import do środowiska Unity}
Po zaimportowaniu modelu do \textit{Unity}, należy wygenerować \textit{collidery}, poprzez zaznaczenie opcji \textit{Generate Colliders} w Inspektorze modelu. Na poszczególnych fragmentach należy zaznaczyć opcję \textit{Convex} w komponencie \textbf{Mesh Collider} w celu umożliwienia wykrywania kolizji pomiędzy tymi obiektami. Aby zminimalizować odbicia obiektu od podłoża po upadku, materiał powinien byś ustawiony na materiał z małym współczynnikiem odbicia, np. metal. Następnie każdemu fragmentowi należy dodać komponent \textbf{RigidBody}, który daje możliwość ustawienia takich cech obiektu jak masa czy opory powietrza wpływające na zachowanie obiektu na scenie. Tak przygotowany obiekt w postaci \textit{prefabu} jest gotowy do użycia na scenie.

\subsubsection{Skrypt}
W celu symulacji zniszczenia obiektu w środowisku \textit{Unity}, konieczne jest przygotowanie modeli przed i po zniszczeniu. W momencie wykrycia działania na obiekt siły wystarczającej do jego zniszczenie, należy podmienić model na scenie. \\
Dla wazy został utworzony skrypt \textit{DesctrucibleVase.cs}, jeżeli w momencie kolizji z innym obiektem zostanie przekroczona uprzednio zdefiniowana wartość progowa, dynamicznie tworzona jest nowa instancji wazy posiadająca za model wazę zniszczoną. Następnie dla każdego fragmentu nowej fazy przypisywana jest rotacja i pęd istniejącego obiektu, po czym obiekt ten jest usuwany ze sceny i zastępowany nowo utworzonym. W ten sposób użytkownik nie widzi podmiany modelu, a jedynie efekt rozpadającej się na kawałki wazy. Ta sama technika może zostać wykorzystana przy dowolnym obiekcie w środowisku \textit{Unity}.

\subsection{Świeca}
Kolejnym elementem prezentowanym na scenie jest świeca. Głównym powodem wyboru tego elementu była potrzeba zaprezentowania ognia, który jest umocowany na przesuwalnym obiekcie. Dzięki temu przesuwając święcę można zaobserwować ruchy płomienia.

\subsubsection{Elementy}
\paragraph{Wosk}
Model części woskowej został stworzony w programie \textit{Blender}. 
c d n....................

\paragraph{Płomień}
Płomień świecy został utworzony w całości w środowisku \textit{Unity}. Do tego celu wykorzystano gotowy element \textit{Particle System}, czyli system cząsteczek. Domyślna konfiguracja tego komponentu nie przypomina ognia, lecz rozproszone cząstki wydobywające się z danego miejsca. Należało więc zmienić właściwości elementu, takie jak prędkość ruchu cząstek, ich maksymalna ilość, kształt grupy cząstek w trakcie cyklu życia czy ich barwa. Płomień został dodany jako podelement do woskowej części świecy, by poruszał się razem z nią.

\paragraph{Źródło światła}