\section{Instrukcja obsługi}

\subsection{Uruchomianie aplikacji}
Aplikacja jest uruchamiana za pomocom pliku o nazwie „DomStrachow.exe”.  Pliki znajdujące się w folderze „data” są niezbędne do jej prawidłowego działania.

\subsection{Ustawienia graficzne}
Aplikacja pozwala na wybór detali wyświetlanych elementów poprzez wybranie przygotowanych zestawów ustawień „ustawienia minimalne”, „ustawienia maksymalne”. Ustawieniem domyślnym jest ustawienie maksymalne.

\subsection{Sterowanie}
Sterowanie postacią odbywa się za pomocą kontrolera Xbox 360 lub za pomocą klawiatury lub myszki według tabeli 1 i 2.

//TODO wstawić tabelę

\begin{table}[H]
	\begin{center}
		\begin{tabular}{|c|c|c|c|c|}
			\cline{3-4}
			\multicolumn{2}{c|}{} & \multicolumn{2}{c|}{Funkcja} \\
			\cline{3-4}
			\multicolumn{2}{c|}{} & Sterowanie postacią & Poruszanie kijem  \\
			\hline
			\multirow{0}{*} Urządzenie & Klawiatura & W, S, A, D lub strzałki & Q, E, R, F  \\
			\cline{2-4}
			& Pad do Xbox360 & Lewa gałka analogowa & Prawa gałka analogowa  \\			
			\hline
		\end{tabular}
	\end{center}
	\caption{Sterowanie}
	\label{ControlTable}
\end{table}


