\section{Instrukcja obsługi}

\subsection{Uruchomianie aplikacji}
Aplikacja jest uruchamiana za pomocom pliku o nazwie „DomStrachow.exe”.  Pliki znajdujące się w folderze „data” są niezbędne do jej prawidłowego działania.

\subsection{Ustawienia graficzne}
Aplikacja pozwala na wybór detali wyświetlanych elementów poprzez wybranie przygotowanych zestawów ustawień „ustawienia minimalne”, „ustawienia maksymalne”. Ustawieniem domyślnym jest ustawienie maksymalne.

\subsection{Sterowanie}
Sterowanie postacią odbywa się za pomocą kontrolera Xbox 360 lub za pomocą klawiatury lub myszki według tabeli 1 i 2.




\begin{table}[H]
	\begin{center}
	\rowcolors{2}{blue!15}{white}   
	\begin{tabular}{|c | c|}
		\rowcolor{blue!50}    
		\multicolumn{2}{|c|}{Sterowanie klawiaturą i myszką}          % Heading with different color to highlight   
	       \\ \hline
		\textbf{Zadanie} & \textbf{Przypisanie} \\
		Rych postacią do przodu & W \\
		Ruch postacią do tyłu & S\\
		Ruch postacią w lewo & A \\
		Ruch postacią w prawo & D  \\
		Obrót postacią & Ruch myszką  \\
		Przesunięcie kija w górę & R  \\
		Przesunięcie kija w dół & F  \\
		Przesunięcie kija w lewo & Q  \\
		Przesunięcie kija w prawo & E  \\
		Podskok & Spacja  \\ \hline
	\end{tabular}
	\end{center}
	\caption{Sterowanie klawiaturą i myszką}
	\label{ControlTable}
\end{table}


\begin{table}[H]
	\begin{center}
		\rowcolors{2}{blue!15}{white}   
		\begin{tabular}{|c | c|}
			\rowcolor{blue!50}    
			\multicolumn{2}{|c|}{Sterowanie klawiaturą i myszką}          % Heading with different color to highlight   
			\\ \hline
			\textbf{Zadanie} & \textbf{Przypisanie} \\
			Rych postacią & Lewy drążek \\
			Obrót postacią & Prawy drążek\\
			Obrót kija & RT + Prawy drążek
		  \\ \hline
		\end{tabular}
	\end{center}
	\caption{Sterowanie kontrolerem Xbox 360}
	\label{ControlTable2}
\end{table}


