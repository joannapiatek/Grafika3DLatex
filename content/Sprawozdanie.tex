%%%%%%%%%%%%%%%%%%%%%%%%%%%%%%%%%%%%%%%%%
% University/School Laboratory Report
% LaTeX Template
% Version 3.1 (25/3/14)
%
% This template has been downloaded from:
% http://www.LaTeXTemplates.com
%
% Original author:
% Linux and Unix Users Group at Virginia Tech Wiki 
% (https://vtluug.org/wiki/Example_LaTeX_chem_lab_report)
%
% License:
% CC BY-NC-SA 3.0 (http://creativecommons.org/licenses/by-nc-sa/3.0/)
%
%%%%%%%%%%%%%%%%%%%%%%%%%%%%%%%%%%%%%%%%%

%----------------------------------------------------------------------------------------
%	PACKAGES AND DOCUMENT CONFIGURATIONS
%----------------------------------------------------------------------------------------

\documentclass{article}

\usepackage{siunitx} % Provides the \SI{}{} and \si{} command for typesetting SI units
\usepackage{graphicx} % Required for the inclusion of images
\usepackage{natbib} % Required to change bibliography style to APA
\usepackage{amsmath} % Required for some math elements 
\usepackage[margin=1.5in]{geometry}


\setlength\parindent{0pt} % Removes all indentation from paragraphs

\renewcommand{\labelenumi}{\alph{enumi}.} % Make numbering in the enumerate environment by letter rather than number (e.g. section 6)

%ustawienie jezyka polskiego
\usepackage{polski}
\usepackage[utf8]{inputenc}
\usepackage[T1]{fontenc}


\graphicspath{ {images/} }


%----------------------------------------------------------------------------------------
%	DOCUMENT INFORMATION
%----------------------------------------------------------------------------------------

\title{Grafika 3D i systemy multimedialne\\
	\vspace{5mm}
	\textbf{POKÓJ STRACHÓW}}

\author{\\
	\\\textbf{Autorzy:}
	\\Piotr Osipa, nr indeksu:
	\\Paweł Andziul, nr indeksu: 
	\\Maciej Kiedrowski, nr indeksu: 200105
	\\Joanna Piątek, nr indeksu: 199966
	\\\\
	\\
	\\\textbf{Grupa:} Czwartek TN/13:15}
\date{\textbf{Data oddania:} 19.01.2017}


%%\date{\today} % Date for the report

\begin{document}

\maketitle % Insert the title, author and date

\begin{center}
\begin{tabular}{l r}
\\\\\\
\textbf{Prowadzący:} & dr inż. Jan Nikodem \\
\\\textbf{Ocena pracy:} &  %
\end{tabular}
\end{center}
 
\newpage
\tableofcontents 	%spis tresci
\newpage

%---------------------------------------------------------------------------

\section{Instrukcja obsługi}

\subsection{Uruchamianie aplikacji}
Aplikacja jest uruchamiana za pomocą pliku o rozszerzeniu .exe.

\subsection{Sterowanie}

\begin{table}[!htbp]
	\centering
	\caption{Sterowanie}
	\label{tab:steering}
	\begin{tabular}{|c|c|c|}
		\hline
		Klawiatura     & Pad do Xbox360  & Opis funkcji \\ \hline
		W, S, A, D lub strzałki  & Lewa gałka analogowa &   Sterowanie postacią \\ \hline
		Q, E, R, F & Prawa gałka analogowa   &  Poruszanie kijem \\ \hline         
	\end{tabular}
\end{table}
\section{Implementacja projektu}

\subsection{Definicja sterowania}
Paweł

\subsection{Sterowanie postacią}
Sterowanie postacią zostało zaimportowane z Standard Assets zawierających m.in. moduł FirstPersonCharacter, w którym zostało zdefiniowane całe sterowanie postacią: kiwanie się kamery przy chodzeniu, dźwięki kroków, postać w kształcie kapsuły, itp.

\subsection{Poruszanie kijem}
Paweł

\subsection{Piłka}
Paweł

\subsection{Wazon i jego rozbicie}
Maciej

\subsection{Świeca}
Joanna


%\newpage
%\section{Bibliografia}

% K. Stąpor
%\textit {Metody klasyfikacji obiektów w wizji komputerowej} Wydawnictwo naukowe PWN, Warszawa 2001

\end{document}